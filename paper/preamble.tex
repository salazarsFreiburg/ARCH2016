% precompiled preamble a la
%
% http://magic.aladdin.cs.cmu.edu/2007/11/02/precompiled-preamble-for-latex/
%
% commands:
%   pdflatex -ini -jobname="main" "&pdflatex preamble.tex\dump"
%   pdflatex -parse-first-line main.tex
%
% - -parse-first-line not needed with TeXlive 2011
% - for latex(1) instead of pdflatex(1), all three occurrences must be
%   replaced
%
%\documentclass[runningheads,orivec]{llncs}
\documentclass{easychair}

\usepackage{xcolor}
\usepackage{amsmath}
\usepackage{amssymb}
\usepackage{mathtools}
\usepackage{float}
\usepackage{tikz}
\usepackage{pgfplots}
\usetikzlibrary{backgrounds,fit,decorations.pathreplacing,snakes,arrows,shapes,automata}
\usetikzlibrary{patterns}
%\usepgfplotslibrary{external}\tikzexternalize
\usepackage{ifthen}
\usepackage{ifpdf}
\usepackage{easpc}
\usepackage{qex}
\usepackage{lskframe}

\ifpdf
\relax
\else
\usepackage{pstricks,pst-node}
\fi





\title{Verification of Fault-Tolerant\\Clock Synchronization Algorithms\\
\normalsize (Benchmark Proposal)}
\author{%
  Sergiy Bogomolov$^{\ddagger}$
  \and
  Christian Herrera$^{\star}$
  \and
  Wilfried Steiner$^{\dagger}$}
\titlerunning{Verification of Fault-Tolerant Clock Synchronization Algorithms}
\authorrunning{Bogomolov, Herrera and Steiner}
\institute{%
  $^{\star}$~Albert-Ludwigs-Universit\"at Freiburg, Freiburg, Germany \\
  $^{\dagger}$~TTTech Computertechnik AG, Chip IP Design, Vienna, Austria \\
  $^{\ddagger}$~IST Austria, Vienna, Austria
} 

% % % % % % % % % % % % % % % % % % % % % % % % % % % % % % % % % % % % % % %

\makeatletter
\newcommand{\todo@box}[1]{\fcolorbox{green!50!black}{green!50!white}{#1}}
\newcommand{\todo}[1]{%
  \begin{center}
  \normalfont%
  \todo@box{\parbox{0.9\textwidth}{\color{black}{\bf TODO}: #1}}
  \end{center}}
\makeatother


% % prelim.tex  % % % % % % % % % % % % % % % % % % % % % % % % % % % % % % %

\newcommand{\ofsane}[1]{\ifthenelse{\equal{#1}{(}}{\errmessage{Hups, '(' as of-macro parameter?}}{}}

\newcommand{\clks}{\mathcal{X}}
\newcommand{\someclks}{X}
\newcommand{\somevars}{V}
\newcommand{\someclksY}{Y}
\newcommand{\clk}{x}
\newcommand{\otherclk}{y}
\newcommand{\thirdclk}{z}
\newcommand{\QE}{\mathcal{QE}}
%
\newcommand{\vars}{\mathcal{V}}
\newcommand{\var}{v}
%
\newcommand{\intexpr}{\psi_\mathit{int}}
%
\newcommand{\Xconstrs}[2][]{\Phi_{#1}(#2)}
\newcommand{\constr}[1][]{\varphi\ifthenelse{\equal{#1}{}}{}{_#1}}
\newcommand{\cconstrs}{\Xconstrs{\clks}}
\newcommand{\cconstr}{\constr_\mathit{clk}}
\newcommand{\vconstrs}{\Xconstrs{\vars}}
\newcommand{\vconstr}{\constr_\mathit{int}}
\newcommand{\constrs}[1][]{\Xconstrs[#1]{\clks,\vars}}
%
\newcommand{\val}{\nu}
\newcommand{\Time}{\mathit{Time}}
%
\newcommand{\clocksofX}[1]{\ofsane{#1}\mathit{clocks}(#1)}
\newcommand{\clocksofta}[1][\ta]{\ofsane{#1}\clks(#1)}
\newcommand{\clocksofrvec}[1][\rvec]{\clocksofX{#1}}
\newcommand{\clocksofconstr}[1][]{\clocksofX{\constr[#1]}}
\newcommand{\varsofX}[1]{\ofsane{#1}\mathit{vars}(#1)}
\newcommand{\varsofrvec}[1][\rvec]{\varsofX{#1}}
\newcommand{\varsofconstr}[1][]{\varsofX{\constr[#1]}}


\newcommand{\locs}{L}
\newcommand{\locsv}{L^{\triangledown}}
\newcommand{\vnstY}[1]{\mathit{vnst}_#1}
\newcommand{\loc}{\ell}
\newcommand{\locv}{\ell_{\triangledown}}
\newcommand{\locva}[1]{\ell_{\triangledown,#1}}
\newcommand{\iloc}[2][]{\loc_{\mathit{ini{#2}\ifthenelse{\equal{#1}{}}{}{#1}}}}
\newcommand{\nstloc}[2][]{\loc_{\mathit{nst{#2}\ifthenelse{\equal{#1}{}}{}{#1}}}}
\newcommand{\tokY}[1][]{\mathit{t}_#1}
\newcommand{\acts}{B}
\newcommand{\comls}{C}
\newcommand{\act}{\alpha}
\newcommand{\locinv}{I}
\newcommand{\edges}{E}
\newcommand{\edge}{e}
\newcommand{\bcacts}{\mathcal{B}}
%
\newcommand{\ta}{\mathcal{A}}
%
\newcommand{\tatupleA}[1][]{%
  (\locs_{#1},\acts_{#1},\clks_{#1},\vars_{#1},}
\newcommand{\tatupleB}[1][]{%
  \locinv_{#1},\edges_{#1},\iloc{})}
\newcommand{\tatuple}[1][]{%
  \tatupleA[#1]\tatupleB[#1]}
%
\newcommand{\rs}{\mathcal{R}(\clks,\vars)}
\newcommand{\rvec}{\vec{r}}
\newcommand{\lvec}{\vec{\loc}}
%
\newcommand{\true}{\mathit{true}}
\newcommand{\false}{\mathit{false}}
%
\newcommand{\edgetuple}[1][]{(\loc_{#1},\act_{#1},\constr_{#1},\rvec_{#1},\loc'_{#1})}
\newcommand{\edgetuplesimpl}{(\loc,\act,\constr,\langle\clk:=0\rangle,\loc')}
\newcommand{\otheredgetuple}[1][]{(\hat{\loc}_{#1},\hat{\act}_{#1},\hat{\constr}_{#1},\hat{\rvec}_{#1},\hat{\loc}'_{#1})}
\newcommand{\otheredgetupleRO}[1][]{(\hat{\loc}_{#1},r_{W}!,\hat{\constr}_{#1},\hat{\rvec}_{#1},\hat{\loc}'_{#1})}
\newcommand{\otheredgetupleRI}[1][]{(\hat{\loc}_{#1},r_{W}?,\hat{\constr}_{#1},\hat{\rvec}_{#1},\hat{\loc}'_{#1})}
\newcommand{\otheredgetupleI}[1][]{(\hat{\loc}_{#1},reset_{W}?,\hat{\constr}_{#1},\hat{\rvec}_{#1},\hat{\loc}'_{#1})}
\newcommand{\otheredgetupleO}[1][]{(\loc_{\resetter{#1}},reset_{#1}!,\hat{\constr}_{#1},\hat{\rvec}_{#1},\nstloc[\resetter{#1}])}
\newcommand{\otheredgetupleR}[1][]{(\ilocR{#1},r_{W}?,\true,\emptyset,\nstlocR{#1})}
%
\newcommand{\locsof}[1]{\ofsane{#1}\locs(#1)}
%\newcommand{\ilocof}[1]{\ofsane{#1}\iloc(#1)}
\newcommand{\edgesof}[1]{\ofsane{#1}\edges(#1)}
\newcommand{\varsof}[1]{\ofsane{#1}\vars(#1)}
\newcommand{\actsof}[1]{\ofsane{#1}\acts(#1)}

\newcommand{\lts}[1]{\ofsane{#1}\mathcal{T}(#1)}
%
\newcommand{\confs}[1]{\ofsane{#1}\mathit{Conf}(#1)}
%\newcommand{\confln}[2]{\langle{#1},{#2}\rangle}
\newcommand{\conf}{c}
\newcommand{\tsconf}{s}
\newcommand{\tsconfdev}{\langle\loc_{\tsconf},\valof{\tsconf}\rangle}
\newcommand{\othertsconf}{r}
\newcommand{\othertsconfdev}{\langle\loc_{\othertsconf},\valof{\othertsconf}\rangle}
%
\newcommand{\iconfs}[1][]{\mathcal{C}_\mathit{ini\ifthenelse{\equal{#1}{}}{}{,#1}}}
%
\newcommand{\labs}{\Lambda}
\newcommand{\lab}{\lambda}
%
\newcommand{\trans}[1]{\xrightarrow{#1}}
\newcommand{\texttrans}[1]{\mathrel{\smash[t]{\trans{#1}}}}
%
\newcommand{\locof}[1]{\ofsane{#1}\loc_{#1}}
\newcommand{\valof}[1]{\ofsane{#1}\val_{#1}}
\newcommand{\confln}[1]{\langle\lvec_{#1}, \valof{#1}\rangle}
%

\newcommand{\comp}{\sigma}
\newcommand{\comps}[1]{\ofsane{#1}\Pi(#1)}
%
\newcommand{\nat}{\mathbb{N}}
\newcommand{\natplus}{\nat^{>0}}
\newcommand{\natnull}{\nat_0}

\newcommand{\parcmp}{\mathrel{\|}}

\newcommand{\nta}{\mathcal{N}}
\newcommand{\mta}{\mathcal{M}}
\newcommand{\sta}{\mathcal{S}}

\newcommand{\res}[1]{\mathcal{RES}_{#1}(\nta)}
\newcommand{\nden}[3][\nta]{\mathcal{DE}^{#2}_{#3}(#1)}
\newcommand{\nde}[3][\ta]{\mathcal{DE}^{#2}_{#3}(#1)}
\newcommand{\lsren}[3][\nta]{\mathcal{LS}^{#2}_{#3}(#1)}
\newcommand{\lcren}[3][\nta]{\mathcal{LC}^{#2}_{#3}(#1)}
\newcommand{\lsre}[2]{\mathcal{LS}^{#1}_{#2}}
\newcommand{\lcre}[2]{\mathcal{LC}^{#1}_{#2}}
\newcommand{\lre}[3][\ta]{\mathcal{LE}^{#2}_{#3}(#1)}
\newcommand{\sre}[1]{\mathcal{SE}_{#1}}
\newcommand{\sren}[3][\nta]{\mathcal{SE}^{#2}_{#3}(#1)}
\newcommand{\cre}[1]{\mathcal{CE}_{#1}}
\newcommand{\cren}[3][\nta]{\mathcal{CE}^{#2}_{#3}(#1)}
\newcommand{\re}[2][]{\mathcal{E}^{#1}_{#2}}
\newcommand{\ren}[3][\nta]{\mathcal{E}^{#2}_{#3}(#1)}
\newcommand{\ul}[1]{\Xi_{#1}}
\newcommand{\congr}[1][\ec]{\equiv_{#1}}
\newcommand{\congrW}[1][W]{\equiv_{#1}}
\newcommand{\hautomaton}{\mathcal{H}}
% % qe.tex  % % % % % % % % % % % % % % % % % % % % % % % % % % % % % % % % %

\newcommand{\qe}{\simeq}

\newcommand{\ecs}[1][\nta]{\mathcal{EC}_{#1}}
%
\newcommand{\ec}{Y}
\newcommand{\ecX}{X}
\newcommand{\otherec}{W}
\newcommand{\rep}{\mathit{rep}}
\newcommand{\seq}[1]{\mathcal{S}_{#1}}
\newcommand{\pt}[1]{\mathcal{P}_{#1}}
\newcommand{\repifynull}{\Gamma_{0}}
\newcommand{\repifyinv}{\Gamma_{\mathit{inv}}}
\newcommand{\repifygrd}{\Gamma}
\newcommand{\repifyct}{\Gamma_{\mathit{c\cdot t}}}

% % wellform.tex  % % % % % % % % % % % % % % % % % % % % % % % % % % % % % %

\newcommand{\Xrloc}[2][]{\mathcal{RL}_{#2}#1}%\ifthenelse{\equal{#2}{\ec}}{(\nta)}{}}
\newcommand{\Nrloc}[2][]{\mathcal{NL}_{#2}#1}%
\newcommand{\rloc}[1]{\Xrloc[^-]{#1}}
\newcommand{\rsuccloc}[1]{\Xrloc[^+]{#1}}

\newcommand{\alg}{\mathcal{K}}
%
\newcommand{\resetter}[1]{\mathcal{R}_{#1}}
\newcommand{\ilocR}[1]{\iloc{\resetter{#1}}}
\newcommand{\locR}[1]{\loc_{\resetter{#1}}}
\newcommand{\nstlocR}[1]{\nstloc{\resetter{#1}}}
\newcommand{\rstX}[2][\ec]{\mathit{rst}_{#1}^{#2}}
\newcommand{\rstI}[1][\ec]{\rstX[#1]{I}}
\newcommand{\rstO}[1][\ec]{\rstX[#1]{O}}
%
\newcommand{\rstIOupd}[1][\edge]{\rho_{#1}}
%
\newcommand{\reset}[1][\ec]{\mathit{reset}_{#1}}
\newcommand{\return}[1][\ec]{\mathit{return}_{#1}}
\newcommand{\rY}[1][\ec]{\mathit{r}_{#1}}
\newcommand{\uY}[1][\ec]{\mathit{u}_{#1}}

% % bisim.tex % % % % % % % % % % % % % % % % % % % % % % % % % % % % % % % %

\newcommand{\sconf}[1]{\mathcal{SC}_{\nta}^{#1}}%\ifthenelse{\equal{#1}{\ec}}{(\nta)}{}}
\newcommand{\dedges}[1]{\mathcal{DE}_{#1}}%\ifthenelse{\equal{#1}{\ec}}{(\nta)}{}}
\newcommand{\othersconf}[1]{\mathcal{SC}_{\nta'}^{#1}}
\newcommand{\Xrseq}[1]{\mathcal{RS}_{#1}}
\newcommand{\rseq}[1]{\Xrseq{#1}}
\newcommand{\rseqpure}[1]{\Xrseq{#1}^\mathit{pure}}
\newcommand{\rseqfullpure}[1]{\Xrseq{#1}^\mathit{full}}
\newcommand{\rseqimpure}[1]{\Xrseq{#1}^\mathit{impure}}

\newcommand{\bigtrans}[1]{\xRightarrow{#1}}
\newcommand{\bigtexttrans}[1]{\mathrel{\smash[t]{\xRightarrow{#1}}}}
%
\newcommand{\Xbsreach}{$\Rightarrow$-reach}
\newcommand{\bsreach}{\Xbsreach{}able}
\newcommand{\bsreaches}{\Xbsreach{}es}

%\newcommand{\BF}{\mathit{BF}}
\newcommand{\BF}{\beta}
\newcommand{\BFof}{\beta}
\newcommand{\CF}{\mathit{CF}}
\newcommand{\EPF}{\mathit{EPF}}
\newcommand{\epfq}[1]{\mathop{\existsOooginool\Diamond} #1}


\newcommand{\wbisimrel}{\mathcal{S}}
\newcommand{\sbisimrel}{\mathcal{S}_{\mathit{str}}}
%\newcommand{\wbisim}{\simeq_\mathit{w}}

\newcommand{\revqe}{\mathit{reverseQE}}
\newcommand{\funqe}{\mathit{QE}}
\newcommand{\devqe}{\mathit{devirQE}}
%
\newcommand{\cons}[1]{\mathit{CONS}_{#1}}
\newcommand{\stable}{\mathit{stable}}
\newcommand{\predelaytrafo}{Z}

% % qeproperties.tex  % % % % % % % % % % % % % % % % % % % % % % % % % % % % % % % % %
\newcommand{\Omg}{\Omega}
\newcommand{\XC}{\mathcal{XC}}
\newcommand{\QC}{\mathcal{QC}}
\newcommand{\CC}{\mathcal{CC}}
\newcommand{\Shi}{\Xi}
\newcommand{\Del}{\Delta}

% % diamond.tex  % % % % % % % % % % % % % % % % % % % % % % % % % % % % % % % % %
\newcommand{\Z}{\mathcal{Z}}
\newcommand{\grph}{G=(V,E)}

% % % % % % % % % % % % % % % % % % % % % % % % % % % % % % % % % % % % % % %

\newcommand{\repval}[4][\val]{#1#3_{#4,\mathit{rep}_{#2}}}
\newcommand{\ECval}[4][\val]{#1#3_{#4,\mathit{QE}_{#2}}}
\newcommand{\nonECval}[4][\val]{#1#3_{#4,\mathit{c}_{#2}}}
\newcommand{\tksval}[2][\val]{#1_{#2,\mathit{tks}}}

\def\preambleloaded{Precompiled preamble loaded.}
